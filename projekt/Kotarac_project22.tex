%% LyX 2.1.3 created this file.  For more info, see http://www.lyx.org/.
%% Do not edit unless you really know what you are doing.
\documentclass[croatian]{article}
\usepackage[T1]{fontenc}
\usepackage{geometry}
\geometry{verbose,tmargin=3cm,bmargin=3cm,lmargin=3cm,rmargin=3cm}
\setlength{\parskip}{\medskipamount}
\setlength{\parindent}{0pt}
\usepackage{amsmath}
\usepackage{amssymb}
\usepackage{graphicx}
\usepackage{setspace}
\usepackage{esint}
\onehalfspacing

\makeatletter
%%%%%%%%%%%%%%%%%%%%%%%%%%%%%% User specified LaTeX commands.
\usepackage{fontspec}
\usepackage{xcolor}
\usepackage{titlesec}
\setmainfont{Droid Sans}

\makeatother

\usepackage{babel}
\usepackage{listings}
\renewcommand{\lstlistingname}{Listing}

\begin{document}

\title{Dizajn šipke sa željenim termičkim svojstvima\\
{\large{}projektni zadatak iz kolegija Znanstveno računanje 2}}


\author{Ilijan Kotarac}

\maketitle

\section{Problem}

Potrebno je dizajnirati metalnu šipku koja je izlivena od legure dva
materijala: \textbf{insulatinuma} i \textbf{conductivituma}. Relativna
gustoća insulatinuma u šipci dana je kao funkcija položaja duž duljine
šipke, $\rho(x).$ Na početnom kraju šipke, za $x=0,$ temperatura
je zadana i iznosi $850K.$ Drugi kraj šipke ne smije prijeći temperaturu
$375K.$

Potrebno je odrediti nepoznatu relativnu gustoću $\rho(x)$ te duljinu
šipke $L$ tako da je trošak izrade minimalan. Jedan centimetar insulatinuma
košta 80\$, a jedan centimetar conductivituma 30\$. Ukupna cijena
šipke dana je izrazom:
\begin{equation}
c=\int_{0}^{L}\left[8\rho(x)+3(1-\rho(x))\right]dx.\label{eq:trosak}
\end{equation}


Distribucija temperature u šipci dana je toplinskom jednadžbom:
\begin{equation}
\kappa(x)u\prime\prime(x)+\kappa\prime(x)u\prime(x)=0,\label{dj}
\end{equation}
gdje je $u$ temperatura šipke na položaju $x,$ a $\kappa(x)$ termička
vodljivost šipke u točki $x$ koja ovisi o njenom sastavu i povezana
je sa gustoćom sljedećom relacijom:
\begin{equation}
\kappa(x)=4\rho(x)^{2}-5\rho(x)+2.\label{kappa}
\end{equation}
Pripadni rubni uvjeti glase:
\begin{eqnarray}
u(0) & = & 850,\label{ru}\\
u\prime(L)+0.01u(L) & = & 0.\nonumber 
\end{eqnarray}
Prvi rubni uvjet označava da se početni kraj šipke drži na temperaturi
$850K,$ a drugi rubni uvjet označava da se na njemu događa apsorpcija
topline.

Dakle, cilj je varirajući nepoznate parametre $\rho(x)$ i $L$ minimizirati
funkciju troška \eqref{eq:trosak} uz ograničenja
\begin{gather*}
0\leq\rho(x)\leq1\\
0\leq L<\infty\\
u(L)\leq375K,
\end{gather*}
pri čemu je zadnji uvjet nelinearan i dobiva se izvrijednjavanjem
rješenja rubnog problema \eqref{kappa} + \eqref{ru}.


\section{Diskretizacija}

Nepoznata funkcija $\rho$ aproksimirana je kubičnim splineom sa $s$
čvorova. Za to je korištena Matlabova funkcija 
\begin{lstlisting}
spline(x, y)
\end{lstlisting}
 gdje su x i y vektori koji definiraju čvorove interpolacije.

\begin{figure}
\noindent \begin{centering}
\includegraphics[width=1\textwidth,bb = 0 0 200 100, draft, type=eps]{skica.pdf}
\par\end{centering}

\protect\caption{Na slici je prikazan učvršćeni štap. $u(x,\,t)$ označava vertikalni
otklon (progib).\label{fig:koordinatni sustav}}
\end{figure}


Uz određene početne uvjete moguće je postići da svaka točka štapa
titra istom frekvencijom, ali amplituda titranja može ovisiti o položaju
$x.$ Takve frekvencije zovu se vlastite frekvencije, a pripadna funkcija
kojom opisujemo transverzalni otklon će u tom slučaju imati oblik:
\[
u(x,\,t)=y(x)e^{i\omega t},
\]
gdje je $\omega$ vlastita (kružna) frekvencija, a $y(x)$ amplituda
titranja koja ovisi o položaju $x.$ Uvrštavanjem u jednadžbu \eqref{eq:PDJ}
dobivamo svojstveni problem:
\begin{equation}
\frac{d^{2}}{dx^{2}}\left(EI\frac{d^{2}y}{dx^{2}}\right)=\rho_{x}\omega^{2}y.\label{eq:Svojstveni problem}
\end{equation}
Iz rubnih uvjeta \eqref{eq:Rubni uvjeti PDJ} dobivamo rubne uvjete
za svojstveni problem \eqref{eq:Svojstveni problem}:
\begin{align}
y(0) & =y(L)=0,\label{eq:Rubni uvjeti za svojstveni problem}\\
y^{\prime\prime}(0) & =y^{\prime\prime}(L)=0.\nonumber 
\end{align}



\section{Homogeni štap konstantnog presjeka}


\subsection{Egzaktno rješenje svojstvenog problema}

Pretpostavimo da štap ima konstantan presjek $S$ duž osi $x,$ i
da je homogen, tako da su veličine $E,\,I$ i $\rho_{x}$ konstantne.
Sada jednadžbu \eqref{eq:Svojstveni problem} možemo pojednostavniti:
\begin{equation}
\frac{d^{4}y}{dx^{4}}=\frac{\rho_{x}}{EI}\omega^{2}y.\label{eq:Svojstveni problem za konstantni presjek}
\end{equation}


Egzaktnim rješavanjem dobivamo svojstvene vrijednosti:
\[
\lambda_{n}=\frac{\rho_{x}}{EI}\omega_{n}^{2}=\left(\frac{n\pi}{L}\right)^{4},\quad n=1,\,2,\,3,\,\dots,
\]
iz čega slijedi formula za svojstvene frekvencije:
\[
\omega_{n}=\sqrt{\frac{EI}{\rho_{x}}}\left(\frac{n\pi}{L}\right)^{2},\quad n=1,\,2,\,3,\,\dots.
\]
Pripadne svojstvene funkcije su:
\begin{equation}
y_{n}(x)=c\sin\left(\frac{n\pi x}{L}\right),\label{eq:egzaktne funkcije}
\end{equation}
gdje je $c\in\mathbb{R}$ proizvoljna konstanta.

Ako na štap djelujemo periodičkom vanjskom silom koja odgovara nekoj
od ovih vlastitih frekvencija, dolazi do pojave rezonancije, sve većih
oscilacija i pucanja štapa. U praksi je dovoljno naći nekoliko najmanjih
svojstvenih frekvencija $\omega_{n},$ jer su ostale prevelike da
bi se prirodno izazvale.


\subsection{Diskretizacija}

Segment $\left[0,\,L\right]$ dijelimo uniformnom mrežom na $N+1$
podintervala i u svakoj točki mreže četvrtu derivaciju aproksimiramo
četvrtom centralnom razlikom. Duljina podintervala iznosi 
\begin{equation}
h=\frac{L}{N+1},\label{eq:duljina podintervala}
\end{equation}
te uvodimo oznaku $f_{\alpha}:=f\left(\alpha h\right).$

Sada su čvorovi:
\[
x_{i}=ih,\quad i=0,\,1,\,\dots,\,N+1.
\]
Približna vrijednost četvrte derivacije dana je četvrtim centralnim
razlikama. U unutrašnjim čvorovima mreže imamo:
\begin{equation}
y_{i}^{\left(4\right)}\approx\frac{y_{i-2}-4y_{i-1}+6y_{i}-4y_{i+1}+y_{i+2}}{h^{4}},\quad i=1,\,\dots,\,N,\label{eq:=00010Cetvrta centralna razlika}
\end{equation}
 pri čemu se javljaju vrijednosti $y_{-1}$ i $y_{N+2}$ u dva fiktivna
vanjska čvora $x_{-1}$ i $x_{N+2}.$

Uvrštavanjem \eqref{eq:=00010Cetvrta centralna razlika} u \eqref{eq:Svojstveni problem za konstantni presjek}
za svaki unutrašnji čvor imamo:
\[
y_{i-2}-4y_{i-1}+6y_{i}-4y_{i+1}+y_{i+2}=\frac{\rho_{x}h^{4}\omega^{2}}{EI}y_{i},\quad i=1,\,\dots,\,N.
\]


Još ostaje eliminirati vrijednosti u fiktivnim čvorovima. Iz rubnih
uvjeta $y(0)=y(L)=0$ imamo $y_{0}=y_{L}=0.$ Rubne uvjete $y^{\prime\prime}(0)=y^{\prime\prime}(L)=0$
ćemo primijeniti nakon što diskretiziramo drugu derivaciju pomoću
druge centralne razlike:
\begin{equation}
y_{i}^{\prime\prime}\approx\frac{y_{i-1}-2y_{i}+y_{i+1}}{h^{2}}.\label{eq:Druga centralna razlika}
\end{equation}
Sada imamo 
\[
y_{0}^{\prime\prime}\approx\frac{y_{-1}-2y_{0}+y_{1}}{h^{2}}=0,\quad y_{N}^{\prime\prime}\approx\frac{y_{N}-2y_{N+1}+y_{N+1}}{h^{2}}=0,
\]
iz čega slijedi:
\[
y_{-1}=-y_{1},\quad y_{N+2}=-y_{N}.
\]


Time smo dobili sljedeći linearni sustav:

\noindent 
\[
\begin{cases}
5y_{1}-4y_{2}+y_{3} & =\frac{\rho_{x}h^{4}\omega^{2}}{EI}y_{1},\\
y_{i-2}-4y_{i-1}+6y_{i}-4y_{i+1}+y_{i+2} & =\frac{\rho_{x}h^{4}\omega^{2}}{EI}y_{i},\quad i=2,\,\dots,\,N-1,\\
y_{N-1}-4y_{N-1}+5y_{N} & =\frac{\rho_{x}h^{4}\omega^{2}}{EI}y_{N},
\end{cases}
\]
što zapisano matrično izgleda ovako:
\[
T(N)\,y=\frac{\rho_{x}h^{4}\omega^{2}}{EI}y,
\]
gdje je 
\[
T(N)=\left[\begin{array}{rrrrrrr}
5 & -4 & 1\\
-4 & 6 & -4 & 1\\
1 & -4 & 6 & -4 & 1\\
 & \ddots & \ddots & \ddots & \ddots & \ddots\\
 &  & 1 & -4 & 6 & -4 & 1\\
 &  &  & 1 & -4 & 6 & -4\\
 &  &  &  & 1 & -4 & 5
\end{array}\right].
\]
Dakle dobili smo algebarski problem svojstvenih vrijednosti.


\subsection{Numeričko rješavanje algebarskog problema svojstvenih vrijednosti.
Kvaliteta aproksimacije svojstvenih frekvencija i funkcija u ovisnosti
o N}

Radi jednostavnosti uzimamo $\rho_{x}=L=E=I=1$ te za zadani $N$
svojstvene vrijednosti i vektore računamo u programskom paketu \texttt{MATLAB}
funkciom \texttt{eig(T(N))}.

Odaberemo li $N$ čvorova, dobivamo matricu dimenzija $N\times N$
koja ima $N$ svojstvenih vrijednosti. Broj $N$ variran je u vrijednostima
iz skupa $\left\{ 10,\,32,\,109,\,329\right\} .$

Naime, želimo postići da se mreže preklapaju u određenim unutrašnjim
čvorovima u kojima ćemo uspoređivati dobivene aproksimativne svojstvene
funkcije. Ako je $N_{1}$ postojeći broj čvorova, a $N_{2}$ broj
čvorova finije razdiobe koji nastaje kad svaki podinterval prve razdiobe
podijelimo na $p$ dijelova, iz \eqref{eq:duljina podintervala} slijedi
sljedeća formula:
\begin{eqnarray*}
h_{1} & = & ph_{2}\\
\frac{L}{N_{1}+1} & = & p\frac{L}{N_{2}+1}\\
N_{2} & = & p(N_{1}+1)-1,
\end{eqnarray*}
pa za $N_{1}=10$ i $p=3,\,10,\,30$ dobivamo $N_{2}=32,\,109,\,329.$

Za svaku vrijednost od $N$ uspoređeno je odstupanje 10 najmanjih
svojstvenih frekvencija i pripadnih 10 svojstvenih funkcija od egzaktnih
vrijednosti.

Usporedba prvih 10 svojstvenih frekvencija sa egzaktnim vrijednostima
za svaki $N$ prikazano je na Slici \ref{fig:10 frekvencija}.

\begin{figure}
\noindent \begin{centering}
\includegraphics[width=1\textwidth,bb = 0 0 200 100, draft, type=eps]{frekvencije_o_N.pdf}
\par\end{centering}

\protect\caption{Prvih 10 svojstvenih frekvencija u ovisnosti o finoći diskretizacije
$N,$ uspoređeno sa prvih 10 egzaktnih svojstvenih vrijednosti.\label{fig:10 frekvencija}}
\end{figure}


Kao mjera odstupanja uzeta je norma odstupanja:
\begin{equation}
\left\Vert \mathbf{\omega}-\mathbf{\omega_{\text{egz}}}\right\Vert _{2},\label{eq:norma odstupanja frekvencija}
\end{equation}
gdje je $\mathbf{\omega}$ vektor koji sadrži 10 najmanjih svojstvenih
frekvencija, a $\mathbf{\omega_{\text{egz}}}$ vektor koji sadrži
10 najmanjih egzaktnih svojstvenih vrijednosti. Ovisnost veličine
\eqref{eq:norma odstupanja frekvencija} o broju $N$ prikazana je
na Slici \ref{fig:ukupno odstupanje}. Apsolutne vrijednosti komponenti
vektora $\mathbf{\omega}-\mathbf{\omega_{\text{egz}}}$ (odstupanje
$\left|\omega_{n}-\omega_{n,\,egz}\right|$u ovisnosti o $n,$ za
različite $N$) prikazane su na Slici \ref{fig:odstupanje po frekvencijama}.
Iz ovoga se zorno vidi da se pri određivanju većih svojstvenih vrijednosti
radi veća greška i da se za manji $N$ također radi veća greška.

\begin{figure}
\noindent \begin{centering}
\includegraphics[width=1\textwidth,bb = 0 0 200 100, draft, type=eps]{odstupanje_ukupno.pdf}
\par\end{centering}

\protect\caption{Ovisnost veličine \eqref{eq:norma odstupanja frekvencija}, kojom
mjerimo kvalitetu aproksimacije prvih 10 svojstvenih vrijednosti,
o finoći diskretizacije $N.$\label{fig:ukupno odstupanje}}
\end{figure}
\begin{figure}
\noindent \begin{centering}
\includegraphics[width=1\textwidth,bb = 0 0 200 100, draft, type=eps]{odstupanje_frekvencija.pdf}
\par\end{centering}

\protect\caption{Odstupanje $\left|\omega_{n}-\omega_{n,\,egz}\right|$u ovisnosti
o $n,$ za različite $N.$\label{fig:odstupanje po frekvencijama}}
\end{figure}


Kvaliteta aproksimacije svojstvenih funkcija mjerena je pomoću matrične
norme
\begin{equation}
\left\Vert \mathbf{V}-\mathbf{V_{\text{egz}}}\right\Vert _{2},\label{eq:odstupanje funkcija}
\end{equation}
gdje je $\mathbf{V}$ matrica tipa $N\times10$ koja sadrži prvih
10 svojstvenih vektora, ali tako da vrijednosti u svakom retku svakog
od vektora odgovaraju vrijednostima funkcije $y$ u 10 istih, ekvidistantnih
čvorova koji prekrivaju čitavi segment $\left[0,\,L\right]$. Na ovaj
način omogućena je usporedba kvalitete dobivenih svojstvenih vektora
za različitu finoću diskretizacije (različit $N$). Također, kako
bi se vektori mogli uspoređivati vektori su normalizirani i ,,naštimani''
tako da budu rastuće funkcije u 1. čvoru. Ukoliko bi u prvom čvoru
bili negativni, pomnoženi su s $-1,$ tako da odgovaraju sinusima
iz relacije \eqref{eq:egzaktne funkcije}.

Ovisnost veličine \eqref{eq:odstupanje funkcija} o broju $N$ prikazana
je na Slici \ref{fig:odstupanje funkcija ukupno}. Usporedba prve
četiri svojstvene funkcije, dobivene za $N=32,$ s egzaktnim rješenjem
prikazana je na Slici \ref{fig:5 svojstvenih funkcija}.

\begin{figure}
\noindent \begin{centering}
\includegraphics[width=1\textwidth,bb = 0 0 200 100, draft, type=eps]{odstupanje_funkcija.pdf}
\par\end{centering}

\protect\caption{Odstupanje $\left\Vert \mathbf{V}-\mathbf{V_{\text{egz}}}\right\Vert _{2},$
u ovisnosti o $N.$\label{fig:odstupanje funkcija ukupno}}
\end{figure}


\begin{figure}
\noindent \begin{centering}
\includegraphics[width=1\textwidth,bb = 0 0 200 100, draft, type=eps]{funkcije_N30.pdf}
\par\end{centering}

\protect\caption{Usporedba prve četiri svojstvene funkcije, dobivene za $N=32,$ $E=I=\rho_{x}=1,$
s egzaktnim rješenjem. \label{fig:5 svojstvenih funkcija}}
\end{figure}



\section{Homogeni štap varijabilnog presjeka}


\subsection{Svojstveni problem}

Sad pretpostavljamo da štap ima pravokutan presjek okomito na centralnu
liniju štapa, ali da je površina tog presjeka $S(x)$ funkcija koordinate
$x.$ Pretpostavimo da štap ima konstantnu širinu $b$ okomito na
ravninu otklona (okomito na ravninu papira na Slici \#\#), a da debljina
$d(x)$, koja leži u ravnini papira, okomito na centralnu liniju,
varira linearno od debljine $d\left(x=0\right)=d_{0},$ do $d\left(x=L\right)=d_{L}.$
Sada je
\begin{equation}
d(x)=d_{0}+\left(d_{L}-d_{0}\right)\frac{x}{L},\quad x\in\left[0,\,L\right],\label{eq:debljina}
\end{equation}
a pripadna površina poprečnog presjeka i moment inercije iznose:
\begin{equation}
S(x)=b\,d(x),\quad I(x)=\frac{1}{12}b\,\left[d(x)\right]^{3}.\label{eq:presjek i inercija}
\end{equation}
Ako dodatno pretpostavimo da je štap homogen, od materijala konstantne
volumne gustoće $\rho$ i Youngovog modula elastičnosti $E$, imamo
izraz za linijsku gustoću štapa:
\[
\rho_{x}(x)=S(x)\rho
\]
te pripadni svojstveni problem:
\begin{equation}
\frac{d^{2}}{dx^{2}}\left(I(x)\frac{d^{2}y}{dx^{2}}\right)=S(x)\frac{\rho}{E}\omega^{2}y.\label{eq:Svojstveni problem za varijabilni presjek}
\end{equation}



\subsection{Diskretizacija}

Jednadžbu \eqref{eq:Svojstveni problem za varijabilni presjek} diskretiziramo
tako da obje druge derivacije aproksimiramo drugom centralnom razlikom
\eqref{eq:Druga centralna razlika} u odgovarajućim čvorovima mreže.
Time dobivamo sljedeći sustav:
\begin{eqnarray*}
\frac{d^{2}}{dx^{2}}\left(I(x)\frac{y_{i-1}-2y_{i}+y_{i+1}}{h^{2}}\right) & = & \frac{\rho\omega^{2}}{E}S_{i}y_{i},\\
\frac{1}{h^{4}}\left[I_{i-1}\left(y_{i-2}-2y_{i-1}+y_{i}\right)-2I_{i}\left(y_{i-1}-2y_{i}+y_{i+1}\right)+\right.\\
\left.+I_{i+1}\left(y_{i}-2y_{i+1}+y_{i+2}\right)\right] & = & \frac{\rho\omega^{2}}{E}S_{i}y_{i},\quad i=1,\,\dots,\,N.
\end{eqnarray*}
Vidimo da se ponovno pojavljuju vrijednosti u fiktivnim čvorovima,
$y_{-1}$ i $y_{N+2}.$ Njih zajedno s vrijednostima u rubnim čvorovima
ponovno eliminiramo koristeći rubne uvjete:
\[
y_{-1}=-y_{1},\quad y_{N+2}=-y_{N},\quad y_{0}=y_{N+1}=0.
\]
Time dolazimo do sljedećeg sustava:

\noindent 
\[
\begin{cases}
-2I_{1}\left(-2y_{1}+y_{2}\right)+I_{2}\left(y_{2}-2y_{3}+y_{4}\right) & =\frac{\rho h^{4}\omega^{2}}{E}S_{1}y_{1},\\
I_{1}\left(-2y_{1}+y_{2}\right)-2I_{2}\left(y_{1}-2y_{2}+y_{3}\right)+I_{3}\left(y_{2}-2y_{3}+y_{4}\right) & =\frac{\rho h^{4}\omega^{2}}{E}S_{2}y_{2},\\
I_{i-1}\left(y_{i-2}-2y_{i-1}+y_{i}\right)-2I_{i}\left(y_{i-1}-2y_{i}+y_{i+1}\right)+I_{i+1}\left(y_{i}-2y_{i+1}+y_{i+2}\right) & =\frac{\rho h^{4}\omega^{2}}{E}S_{i}y_{i},\quad i=3,\,\dots,\,N-2,\\
I_{N-2}\left(y_{N-3}-2y_{N-2}+y_{N-1}\right)-2I_{N-1}\left(y_{N-2}-2y_{N-1}+y_{N}\right)+I_{N}\left(y_{N-1}-2y_{N}\right) & =\frac{\rho h^{4}\omega^{2}}{E}S_{N-1}y_{N-1},\\
I_{N-1}\left(y_{N-2}-2y_{N-1}+y_{N}\right)-2I_{N}\left(y_{N-1}-2y_{N}\right) & =\frac{\rho_{x}h^{4}\omega^{2}}{EI}y_{N},
\end{cases}
\]
što se matrično može zapisati na način:
\begin{equation}
\underbrace{\left[\begin{array}{rrrrr}
-2I_{1} & I_{2}\\
I_{1} & -2I_{2} & I_{3}\\
 & \ddots & \ddots & \ddots\\
 &  & I_{N-2} & -2I_{N-1} & I_{N}\\
 &  &  & I_{N-1} & -2I_{N}
\end{array}\right]}_{M_{1}}\underbrace{\left[\begin{array}{rrrrr}
-2 & 1\\
1 & -2 & 1\\
 & \ddots & \ddots & \ddots\\
 &  & 1 & -2 & 1\\
 &  &  & 1 & -2
\end{array}\right]}_{M_{2}}y=\frac{\rho h^{4}\omega^{2}}{E}\underbrace{\left[\begin{array}{rrr}
S_{1}\\
 & \ddots\\
 &  & S_{N}
\end{array}\right]}_{M_{3}}y.\label{eq:Algebarski svojstveni problem za varijabilnu debljinu}
\end{equation}
Pri tome se vrijednosti $I_{i}$ i $S_{i}$, pomoću formula \eqref{eq:debljina}
i \eqref{eq:presjek i inercija} računaju na na način:
\begin{align}
S_{i} & =b\left(d_{0}+\left(d_{L}-d_{0}\right)\frac{ih}{L}\right)\overset{\eqref{eq:duljina podintervala}}{=}b\left(d_{0}+\left(d_{L}-d_{0}\right)\frac{i}{N+1}\right),\label{eq:Presjek i inercija u =00010Dvorovima}\\
I_{i} & =\frac{1}{12}b\,\left[d(x)\right]^{3}=\frac{S_{i}^{3}}{12b^{2}},\quad i=1,\,\dots,\,N.\nonumber 
\end{align}
Iz jednadžbi \eqref{eq:Presjek i inercija u =00010Dvorovima} vidi
se da će matrica $M_{3},$ u slučaju $d_{L}\neq d_{0}$ imati sve
različite vrijednosti na dijagonali, pa će biti regularna i njen inverz
je dan formulom:
\[
M_{3}^{-1}=\left[\begin{array}{rrr}
\frac{1}{S_{1}}\\
 & \ddots\\
 &  & \frac{1}{S_{N}}
\end{array}\right].
\]
 U slučaju $d_{L}=d_{0}$ imamo identitetu, $M_{3}=bd_{0}I_{N}.$
Množenjem jednadžbe \eqref{eq:Algebarski svojstveni problem za varijabilnu debljinu}
s $M_{3}^{-1}$ slijeva dobivamo sljedeći algebarski problem svojstvenih
vrijednosti:
\begin{equation}
\underbrace{\left(M_{3}^{-1}M_{1}M_{2}\right)}_{T_{1}(N)}y=\frac{\rho h^{4}\omega^{2}}{E}y\label{eq:svojstveni diskretni varijablina}
\end{equation}



\subsection{Numeričko rješavanje algebarskog problema svojstvenih vrijednosti}

Prvih 10 dobivenih svojstvenih vrijednosti i prvih 5 svojstvenih funkcija
prikazane su na Slikama \ref{fig:sv frekv var} i \ref{fig:sv funkc var},
redom. Radi jednostavnosti su korišteni podaci $E=\rho=1,$ $L=1,$
$b=0.1,$ $d_{0}=0.01,$ $d_{L}=0.03,$ iz čega je pomoću \eqref{eq:svojstveni diskretni varijablina}
izračunata $\omega.$

\begin{figure}
\noindent \begin{centering}
\includegraphics[width=1\textwidth,bb = 0 0 200 100, draft, type=eps]{frekvencije_var.pdf}
\par\end{centering}

\protect\caption{Prvih deset svojstvenih frekvencija u ovisnosti o $n,$ za $N=100.$\label{fig:sv frekv var}}
\end{figure}
\begin{figure}
\noindent \begin{centering}
\includegraphics[width=1\textwidth,bb = 0 0 200 100, draft, type=eps]{funkcije_var.pdf}
\par\end{centering}

\protect\caption{Prvih pet svojstvenih funkcija, dobivenih za $N=100,$ $E=\rho=1,$
$L=1,$ $b=0.1,$ $d_{0}=0.01,$ $d_{L}=0.03$. \label{fig:sv funkc var}}
\end{figure}

\end{document}
